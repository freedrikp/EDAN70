\documentclass{beamer}
\usepackage[utf8]{inputenc}
\usepackage[english]{babel}
%\usepackage{moreverb}
%\usepackage{graphicx}
\usetheme{Madrid}
\title[Classifying Laser Range Data "Images"] % (optional, only for long titles)
{Classifying Laser Range Data "Images"}
%\subtitle{}
\author[F. Paulsson, S. Senanayake]% (optional, for multiple authors)
{Fredrik Paulsson \and Shan Senanayake}
\institute[LTH] % (optional)
{Lund University \\ Faculty of Engineering}
\date[\today] % (optional)
{\today}
%\subject{Computer Science}
%\setcounter{secnumdepth}{5}
%\setcounter{tocdepth}{5}
\begin{document}
\frame{\titlepage}
%\tableofcontents

\begin{frame}
\frametitle{Project Description}
\framesubtitle{Purpose and Goals}

The main task of the project was to make a robot identify certain elements in its surroundings, for example, doors, chairs, tables, humans, e.t.c.

\vspace{10pt}

Thus we divided the project into three parts:

\begin{itemize}
  \item{Create a parser for the laser range data and transform the data into a more suitable coordinate system}
  \item{Develop an algorithm that can classify laser range measurements}
  \item{If time is available, port the programs into ROS (Robot Operating System}
\end{itemize}

Offline data to be used during development and if ported into ROS maybe test on online data

\end{frame}



\end{document}