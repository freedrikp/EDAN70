\documentclass{beamer}
\usepackage[utf8]{inputenc}
\usepackage[english]{babel}
\usepackage{moreverb}
%\usepackage{graphicx}
\usetheme{Madrid}
\title[Classifying Laser Range Data "Images"] % (optional, only for long titles)
{Classifying Laser Range Data "Images"}
\subtitle{Supervisor: Elin Anna Topp}
\author[F. Paulsson, S. Senanayake]% (optional, for multiple authors)
{Fredrik Paulsson \and Shan Senanayake}
\institute[LTH] % (optional)
{Lund University \\ Faculty of Engineering}
\date[\today] % (optional)
{\today}
%\subject{Computer Science}
%\setcounter{secnumdepth}{5}
%\setcounter{tocdepth}{5}
\begin{document}
\frame{\titlepage}
%\tableofcontents

\begin{frame}
\frametitle{Project Description}
\framesubtitle{Purpose and Goals}

The main task of the project was to make a robot identify certain elements in its surroundings, for example, doors, chairs, tables, humans, e.t.c.

\vspace{10pt}

Thus we divided the project into three parts:

\begin{itemize}
\pause
\item{Create a parser for the laser range data and transform the data into a more suitable coordinate system}
\pause  
\item{Develop an algorithm that can classify laser range measurements}
\pause  
\item{If time is available, port the programs into ROS (Robot Operating System)}
\end{itemize}
\pause
Offline data to be used during development and if ported into ROS maybe test on online data

\end{frame}

\begin{frame}[fragile]
\frametitle{Laser Range Data}
\framesubtitle{Parser}

The laser scanner took measurements with 4-5 Hz. Each measurement was stored in the following format:
\begin{verbatim}
<unknown> <unknown> <number_of_points> <timestamp_seconds>
<timestamp_microseconds> <unknown> <unknown> <unknown>
<unknown> <unknown> <unknown> <angle_between_points>
<unknown> <unknown> <unknown>
<distance_in_meters>^(number_of_points)
\end{verbatim}

1 measurement = 1 line
\\
1 round of measurements = 1 file

\pause
\vspace{10pt}

The parser yields polar coordinates which are hard to work with.

\end{frame}

\begin{frame}
\frametitle{Laser Range Data}
\framesubtitle{Transformation}

When we have parsed the measurements we transform the data from polar coordinates to our own cartesian-coordinates.

\pause
\vspace{10pt}

The robot is located at (0,0) and the robot forward is along the y-axis.

\pause
\vspace{10pt}

Having this coordinate-system makes it much easier to plot the data for us to visually analyze the measurements as well as to process the data.

\pause
\vspace{10pt}

For instance, we can use well-known algorithms and mathematical formulas to interpolate the data or find patterns in the data.


\end{frame}

\begin{frame}
\frametitle{Laser Range Data}
\framesubtitle{Example Plots}

INSERT 4 PLOTS ON THIS FRAME


\end{frame}


\begin{frame}
\frametitle{Algorithm}
\framesubtitle{Angle of Attack}

We are both very interested in the Machine Learning division of Artificial Intelligence.

\pause
\vspace{10pt}

Therefore, our original vision was to try to implement the classifier using unsupervised learning. Thereby letting the classifier improve itself.

\pause
\vspace{10pt}

However, we eventually realized that we needed to use supervised learning in order to have a starting ground.


\pause
\vspace{10pt}

Eventually we had to only use supervised learning as we had to change the way the whole classifier was going to work.





\end{frame}


\end{document}